\documentclass[toc=flat,fontsize=11pt,a4paper,titlepage,headsepline,numbers=noenddot, bibliography=totoc]{scrartcl}
%\setcounter{secnumdepth}{5}
%\setcounter{tocdepth}{5}
\usepackage[utf8]{inputenc}
\usepackage[ngerman]{babel} 

\begin{document}

In welcher APDU wird die PIN übertragen?
Sehr wahrscheinlich verschlüsselt, mehr als wahrscheinlich
Antwort: in keiner; die PIN wird nur lokal in der Software verwendet


Erweiterung VirtualSmartcard:
- RESET darf für RelayOS in der Methode RUN nicht stattfinden, da auf das Attribut self.MF
	nicht zugegriffen werden kann. => Anpassung in Methode \verb+vicc->run+

\begin{table}[h] % Platzierung
 \caption{APDUs}
 \begin{tabular}{ccc}
  00 & A4 & Select File \\
  00 & CA & Get Data Plain\\
  0C & 22 & Manage Security Environment - MSE\\
  0C & 2A & perform security operation
 \end{tabular}
 \label{tab:meinetabelle}
 \end{table}
 0F800A04007F00070202040202830102
 
 Bachelorarbeit teildynamische PIN
 Pace
 Abfrage der AusweisApp        CLA INS P1 P2 Lc Data Field              Le
Select MF                     00   A4 00 0C 02 3F 00
00A4000C023F00
Select EF_CARDACCESS          00   A4 02 0C 02 01 1C
Read Binary                   00   B0 00 00                            80
Read Binary(short identifier)  00  B0 9C 00                            00
MSE:AT                        00   22 C1 A4 0F 80 0A ...83 01 password(01|02|03)
GA: Get Nonce                  10  86 00 00 02 7C 00                   00
GA: Map Nonce                  10  86 00 00 45 7C 43 81 41 EC Point    00
GA: Perform Key Agreement      10  86 00 00 45 7C 43 83 41 EC Point    00
GA: Mutual Authentication      00  86 00 00 0C 7C 0A 85 08 MAC         00
Reset Retry Counter           00   2C 02 03 06 new PIN                 00

CHAT vom Beispiel 

7F 4C CHAT Länge 12 
06 - Objektidentifikation  Länge: 09 Wert: 04 00 7F 00 07 03 01 02 02(02 steht für Authentisierungsterminal) 
53 - TAG für Zugriffsrechte Länge: 05 Wert:  00 00 00 81 00

andere Beispiele

MSE inkl. CHAT
0022C1A40F800A04007F00070202040202830102 CAN
0022C1A40F800A04007F00070202040202830103 PIN
- Vornamen auslesen 
APDU: 0022
		C1A4
			24800A04007F000702020402028301 03 Passwort: PIN
		7F4C
			12
			06 09 04007F0007030102 02 - Rolle: Authentisierungsterminal
			53 05 0000000800
			
- Alterstest + Geburtsdatum auslesen
APDU: 0022
		C1A4
			24800A04007F000702020402028301 03 Passwort: PIN
		7F4C
			12
			06 09 04007F0007030102 02 - Rolle: Authentisierungsterminal
			53 05 0000008001
Beim Alterstest ist das letzte Bit im Datenfeld des OI 53 gesetzt


Ausführlich in BSI TR-03110-3

Beispiel: NPA-TOOL - Pin ändern

Die Gegenstelle vom MRTD bei PACE ist hier das NPA-TOOL 
1. APDU: 00 22 C1 A4 0F 80 0A 04 00 7F 00 07 02 02 04 02 02 83 01 03 - Pace starten (MSE inkl. CHAT)
2. APDU: 10 86 00 00 02 7C 00 00
3. Eingabe der alten und anschließend der neuen PIN in der Shell
4. APDU: 10 86 00 00 45 7C 43 81 41 04 0D 1F 53 17 A7 10 E9 65 7F 29 C4 E6 D2 56 E0 F8 30 56 31 42 D3 7E 36 67 39 97 5E 6B 97 37 CE 7A 93 A2 8D B1 CE 3B 07 24 91 83 E9 C0 31 28 01 38 0B 37 88 44 10 75 76 A8 5F A5 6E 39 43 AA EF 31 00
5. APDU: 10 86 00 00 45 7C 43 83 41 04 69 86 22 E8 CF 40 6A 6A B5 FC 9F 43 12 5D 61 CE ED 48 F6 D0 23 D9 67 EC 8B 51 87 DE 1D 88 7D 45 7B A4 B6 06 5E A8 94 9A 84 3E AA 30 8D 17 BF 0A 09 19 C6 E9 D8 76 C4 53 30 E0 B7 D4 7A 60 FB 02 00
6. APDU: 00 86 00 00 0C 7C 0A 85 08 36 90 6B A5 41 47 DA 96 00
7. APDU: 0C 2C 02 03 1D 87 11 01 DF 8A 93 25 03 F7 FF AA 0D 2A 74 19 3E 1E 7A A5 8E 08 AA DD BF 7C 39 A0 FD AE 00 - Kommando zum Ändern der PIN an die Karte

AusweisApp
Init 
- ohne A4 * gleich B0 9C (read binary)
- FF 9A 01 01 00 -> Reiner SCT
- FF 9A 01 03 00
- FF 9A 01 06 00
- FF 9A 01 07 00
- ControlCode 0x42000D48 - GET_FEATURE_REQUEST

Zur Ermittlung, ob es sich um einen nPA handelt, wird nur das EF.DIR 2F00 (Liste der Kartenapplikationen) ausgelesen - 00B00000FF


\end{document}